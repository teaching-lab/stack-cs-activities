%&program=xelatex
%&encoding=UTF-8 Unicode
%§
\documentclass[a4paper,12pt]{article}
\usepackage{fontspec}
\usepackage{xunicode}
\usepackage{xltxtra}
\usepackage{unicode-math}
\usepackage{polyglossia}
%§
\setmainlanguage{czech}
\setotherlanguage[variant=british]{english}
%§
\usepackage{hyperref}
%§
\def\uv#1{„#1“}
%§
\defaultfontfeatures{Scale=MatchLowercase,
                     Mapping=tex-text}
\setromanfont{DejaVu Serif}
\setsansfont{DejaVu Sans}
\setmonofont{DejaVu Sans Mono}
\setmathfont{Asana Math}
%§
\begin{document}
\textenglish{Hello, World!}

%§
V~této větě by měla být použita nezlomitelná mezera~-- do 
zdrojového textu se vkládá jako znak tilda (\char`\~).

Text vložíme do „uvozovek“.

%§
\v{S}\'{\i}len\v{e} \v{z}lu\v{t}ou\v{c}k\'{y} k\r{u}\v{n}
\'{u}p\v{e}l \v{d}\'{a}belsk\'{e} \'{o}dy.

%§
Je snadné vybrat {\fontspec{TeX Gyre Chorus} jiný font}, 
použít barvičky (%
{\addfontfeature{Colour=FF0000}R}%
{\addfontfeature{Colour=00FF00}G}%
{\addfontfeature{Colour=0000FF}B}%
) apod.

%§
\end{document}
% kódování souboru: UTF-8
% engine: xelatex
