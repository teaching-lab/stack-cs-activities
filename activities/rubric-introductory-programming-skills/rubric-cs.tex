\documentclass{article}
\usepackage[czech]{babel}			% Language
\usepackage[utf8]{inputenc}			% Encoding of characters in this .tex file
\usepackage{cmap}					% Make PDF file searchable and copyable (ASCII characters)
\usepackage{lmodern}				% Make PDF file searchable and copyable (Accented characters)
\usepackage[T1]{fontenc}			% Hyphenate accented words
\usepackage[landscape, margin=1.5cm]{geometry}	% Paper size and margins
\usepackage[protrusion]{microtype}	% Better typeset results
\usepackage{float}

\pagenumbering{gobble}
\def\arraystretch{1.5}

\begin{document}
\begin{table}[H]
\centering
\begin{tabular}{|l|p{2.5cm}|l|l|p{4.5cm}|l|l|p{4.5cm}|l|l|p{4.5cm}|}
\hline
& \textbf{Neznalý(á)} & & & \textbf{Začátečník} & & & \textbf{Absolvent kurzu} & & & \textbf{Ideál (vize do budoucna)} \\ \hline

\textbf{1. Algoritmizace}
& Nevím, co je algoritmus nebo je vůbec neumím navrhovat. & &
& S dostatkem času anebo s pomocí dokážu navrhnout jednoduchý algoritmus. Často ale udělám chybu nebo se zaseknu. & & 
& Dokážu algoritmicky myslet a vyjádřit kroky algoritmu v~kódu. & & 
& Umím vymyslet a pochopit různé způsoby řešení problému a porovnat výhody, nevýhody a efektivitu jednotlivých přístupů. CLRS je moje oblíbená kniha. \\ \hline

\textbf{2. Datové struktury}
& Neznám nebo neumím používat datové struktury. & &
& Znám základní datové struktury jako je seznam a zkusil(a) jsem je použít. & & 
& Znám a umím použít seznam, slovník, a jednoduché objekty v Pythonu. & & 
& Detailně znám implementaci a způsob reprezentace datových struktur, rozumím časové a paměťové složitosti jednotlivých operací a beru je v potaz při programování. \\ \hline

\textbf{3. Funkce}
& Nevím dělit kód do funkcí. & &
& Dokážu napsat jednoduchou funkci bez návratové hodnoty. & & 
& Dokážu rozdělit problém do několika samostatných funkcí, které vzájemné využívají své návratové hodnoty. & & 
& Vhodně člením celý program do funkcí, můj kód je dobře čitelný. Moje funkce mají jedinou zodpovědnost a jasný kontrakt. \\ \hline

\textbf{4. Práce s~chybama}
& Bojím se chyb v kódu. Nedokážu opravit kód v případě chyby. & &
& Dokážu opravit jednoduché syntaktické chyby (například chybějící závorka), ale s jinými chybami obvykle potřebuju pomoc. & & 
& Rozumím chybovým hláškám v Pythonu a dokážu samostatně opravit běžné syntaktické, sémantické i logické chyby. & & 
& Dokážu porozumět i cizímu kódu. I v případě složitých problémů umím pomoct kolegům/spolužákům. \\ \hline

\textbf{5. Dokumentace}
& Nikdy jsem nečetl(a) oficiální dokumentaci. & &
& Už jsem četl(a) oficiální dokumentaci k Pythonu nebo k nějakému jeho modulu, ale občas se v ní těžko orientuju. & & 
& Pokud nerozumím nějaké funkci v Pythonu, najdu si k ní oficiální dokumentaci, přečtu si ji a na základě toho pochopím použití funkce. & & 
& Vím, jaká kritéria má splňovat kvalitní dokumentace a už jsem i nějakou napsal(a). \\ \hline
\end{tabular}
\end{table}

\begin{center}
\small Autor: Valdemar Švábenský, verze: 16.7.2018
\end{center}
\end{document}
